\documentclass{beamer}

\usepackage{polyglossia}
\usepackage{fontspec}
\usepackage{nameref}
\usepackage{ifthen}

\usefonttheme{professionalfonts}
\usetheme{Antibes}
\useoutertheme{infolines_foot}
\setbeamercovered{transparent=20}

\usepackage[math-style=ISO,vargreek-shape=unicode]{unicode-math}

\setdefaultlanguage[spelling=modern,babelshorthands=true]{russian}
\setotherlanguage{english}

\defaultfontfeatures{Ligatures={TeX}}
\setmainfont{CMU Serif}
\setsansfont{CMU Sans Serif}
\setmonofont{CMU Typewriter Text}
\setmathfont{Latin Modern Math}
\AtBeginDocument{\renewcommand{\setminus}{\mathbin{\backslash}}}

\makeatletter
\newcommand*{\currentname}{\@currentlabelname}
\makeatother
\def\t{\texttt}

\newcommand{\cimg}[2]{%
	\begin{center}%
		\ifthenelse{\equal{#2}{}}{%
			\includegraphics[width=0.75\linewidth]{#1}
		}{%
			\includegraphics[width=#2\linewidth]{#1}
		}%
	\end{center}%
}

\title[Граф по ридам ДНК/РНК]{Построение графа связей геномных последовательностей по ридам ДНК/РНК}
\author{Черникова Ольга, Щербин Егор}
\institute[СПб АУ РАН]{Руководитель: Пржибельский Андрей \\ СПб АУ РАН }
\date{Весна 2016}

\begin{document}

\begin{frame}
	\titlepage
\end{frame}

\section{Постановка задачи}

\begin{frame}[t]{Проблема сборки генома}
    \begin{itemize}
    \item
        Одной из задач биоинформатики является задача сборки генома.
    \item
        Специальная программа (ассемблер) собирает геном организма по коротким
        его участкам (ридам).
    \item
        К сожалению, собрать геном полностью не получается. Вместо этого
        ассемблер выдаёт несколько достаточно больших его частей (контигов).
    \end{itemize}
\end{frame}

\begin{frame}[t]{Цель проекта}
    \begin{itemize}
    \item
        Наша задача "--- установить относительный порядок между контигами.
    \item
        Порядок визуализируется в виде ориентированного
        графа, в котором вершины "--- контиги, а рёбра "--- вероятные связи
        между ними.
    \end{itemize}

    \cimg{1.jpg}{1}

\end{frame}

\begin{frame}[t]{Подзадачи}
	\begin{enumerate}
    \item
        Разобраться с необходимой теорией и существующеми средствами обработки
        данных.
    \item
        Выбрать вспомогательную библиотеку биоинформатических алгоритмов.
    \item
        Реализовать построение графа несколькими разными способами:
        
        \begin{itemize}
        \item
            По ридам ДНК
        \item
            По ридам РНК
        \end{itemize}
    \item
        Проанализировать получившийся результат.
	\end{enumerate}
\end{frame}

\section{Реализация}
\subsection{ДНК}

\begin{frame}[t]{Парные риды}
    \begin{itemize}
    \item
        Современные технологии позволяют считывать риды парами на небольшом
        расстоянии друг от друга.
    \item
        Если риды из одной пары попали на разные контиги, то, вероятно, эти 
        контиги идут друг за другом.
    \item
        Когда таких признаков становится достаточно много, проводим ребро.
    \end{itemize}
    \cimg{2.jpg}{0.75}
\end{frame}

\begin{frame}[t]{Фильтрация}
    \begin{itemize}
    \item
        Слишком короткие контиги удаляются из графа.
    \item
        Пара не учитывается, если расстояние между ридами слишком большое.
    \item
        Ребро проводится, если его вес преодолел некоторый порог.
    \item
        Для определения порога строится гистограмма весов.
    \end{itemize}
\end{frame}

\subsection{РНК}

\begin{frame}[t]{Интроны и экзоны}
    \begin{itemize}
    \item
        РНК получается из ДНК в результате транскрипции. В процессе сплайсинга
        в РНК остаются лишь некоторые фрагменты исходной молекулы ДНК.
    \item
        С помощью этой информации также можно строить граф.
    \end{itemize}
    \cimg{3.jpg}{0.75}
\end{frame}

\begin{frame}[t]{Как проводится ребро}
    \begin{itemize}
    \item
        Если транскрипт выравнивается на два контига, то между ними проводится
        ребро.
    \item
        Но транскрипты, как и геном, неизвестны, поэтому ищем риды РНК,
        выравнивающиеся на разные контиги.
    \end{itemize}

    \cimg{4.jpg}{0.6}
\end{frame}

\begin{frame}[t]{Проблема с частичным выравниванием}
    \begin{itemize}
    \item
        Современные программы не поддерживают частичное выравнивание коротких 
        ридов.
    \item
        Поэтому было решено сначала собрать риды ассемблером, и выравнивать
        уже длинные части РНК.
    \end{itemize}

    \cimg{4.jpg}{0.6}
\end{frame}

\begin{frame}[t]{Фильтрация}
    \begin{itemize}
    \item
        Слишком короткие контиги убираются из графа.
    \item
        Фильтрация по длине выравнивания.
    \item
        Фильтрация по качеству выравнивания.
    \end{itemize}

    \cimg{4.jpg}{0.6}
\end{frame}

\section{Итоги}
\begin{frame}[t]{Что получилось}
    Парные риды ДНК:
    \cimg{5.jpg}{1}
    РНК:
    \cimg{6.png}{0.5}
\end{frame}

\begin{frame}[t]{Чему мы научились}
    \begin{enumerate}
    \item
        Научились применять различные программы для анализа геномных и
        транскриптомных данных (bowtie2, STAR, nucmer, SPAdes).
    \item
        Освоили библиотеку SeqAn.
    \item
        Обучились SSH для работы на удалённом сервере.
    \item
        Разобрались с SAM-форматом для хранения результатов выравнивания и 
        dot-форматом для визуализации графов.
    \item
        Получили опыт работы в командном научном проекте.
    \end{enumerate}
\end{frame}

\begin{frame}[t]{Возможные пути развития}
    \begin{itemize}
    \item
        Дальнейшее исследование построения графа по ридам РНК.
    \item
        Объединение результатов обоих методов построения в один граф.
    \item
        Автоматизация подборки порогов фильтрации.
    \item
        Изучить другие идеи решения поставленной задачи, например, 
        \texttt{l\_rna\_scaffolder}.
    \end{itemize}
\end{frame}

\section{Спасибо за внимание}
\begin{frame}{Спасибо за внимание}
    \begin{center}
        Репозиторий: \\ \url{https://github.com/eshcherbin/spbau-bioinf-2016}
    \end{center}
\end{frame}
\end{document}
