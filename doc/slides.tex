\documentclass{beamer}

\usepackage{polyglossia}
\usepackage{fontspec}
\usepackage{nameref}
\usepackage{ifthen}

\usefonttheme{professionalfonts}
\usetheme{Antibes}
\useoutertheme{infolines_foot}
\setbeamercovered{transparent=20}

\usepackage[math-style=ISO,vargreek-shape=unicode]{unicode-math}

\setdefaultlanguage[spelling=modern,babelshorthands=true]{russian}
\setotherlanguage{english}

\defaultfontfeatures{Ligatures={TeX}}
\setmainfont{CMU Serif}
\setsansfont{CMU Sans Serif}
\setmonofont{CMU Typewriter Text}
\setmathfont{Latin Modern Math}
\AtBeginDocument{\renewcommand{\setminus}{\mathbin{\backslash}}}

\makeatletter
\newcommand*{\currentname}{\@currentlabelname}
\makeatother
\def\t{\texttt}

\newcommand{\cimg}[2]{%
	\begin{center}%
		\ifthenelse{\equal{#2}{}}{%
			\includegraphics[width=0.75\linewidth]{#1}
		}{%
			\includegraphics[width=#2\linewidth]{#1}
		}%
	\end{center}%
}

\title[Граф по ридам ДНК/РНК]{Построение графа связей геномных последовательностей по ридам ДНК/РНК}
\author{Черникова Ольга, Щербин Егор}
\institute{Руководитель: Пржибельский Андрей \\ СПб АУ РАН }
\date{Весна 2016}

\begin{document}

\begin{frame}
	\titlepage
\end{frame}

\section{Постановка задачи}

\begin{frame}[t]{Проблема сборки генома}
    \begin{itemize}
    \item
        Одной из задач биоинформатики является задача сборки генома.
    \item
        Специальная программа (ассемблер) собирает геном организма по коротким
        его участкам (ридам).
    \item
        К сожалению, собрать геном полностью не получается. Вместо этого
        ассемблер выдаёт несколько достаточно больших его частей.
    \end{itemize}
\end{frame}

\begin{frame}[t]{Цель проекта}
    \begin{itemize}
    \item
        Наша задача "--- установить относительный порядок между этими частями.
    \item
        А именно, построить ориентированный граф, такой что если данные 
        указывают, что одна из них идёт перед второй, то между ними проведено
        ребро.
    \end{itemize}

    \cimg{1.jpg}{1}

\end{frame}

\begin{frame}[t]{Подзадачи}
	\begin{enumerate}
    \item
        Разобраться с необходимой теорией и существующеми средствами обработки
        данных.
    \item
        Выбрать вспомогательную библиотеку биоинформатических алгоритмов.
    \item
        Реализовать построение графа несколькими разными способами (с помощью 
        ридов ДНК и РНК).
    \item
        Проанализировать и улучшить результат.
	\end{enumerate}
\end{frame}

\section{Реализация}
\subsection{ДНК}

\begin{frame}[t]{Парные риды}
    \begin{itemize}
    \item
        Современные технологии позволяют считывать риды парами на небольшом
        расстоянии друг от друга.
    \item
        Если риды из одной пары попали на разные части, то, вероятно, эти части
        идут друг за другом.
    \item
        Когда таких признаков становится достаточно много, проводим ребро.
    \end{itemize}
    \cimg{2.jpg}{0.75}
\end{frame}

\begin{frame}[t]{Фильтрация}
    \begin{itemize}
    \item
        Проводим ребро, если вес преодолел некоторый порог.
    \item
        Удаляем слишком короткие части из графа.
    \item
        Если расстояние между ридами слишком большое, не учитываем эту пару.
    \item
        Построение гистограммы весов для определения порога.
    \end{itemize}
\end{frame}

\subsection{РНК}

\begin{frame}[t]{Интроны и экзоны}
    \begin{itemize}
    \item
        При транскрипции РНК получается из ДНК не полностью, некоторые фрагменты 
        удаляются.
    \item
        В итоге рид РНК разбивается в ДНК на несколько разрозненных кусочков.
    \item
        С их помощью можно строить граф так же, как с парными ридами.
    \end{itemize}
    \cimg{3.jpg}{0.75}
\end{frame}

\begin{frame}[t]{Проблема с частичным выравниванием}
    \begin{itemize}
    \item
        К сожалению, используемые программы не поддерживают кусочное 
        выравнивание коротких ридов.
    \item
        Поэтому было решено сначала собрать эти риды ассемблером, и выравнивать
        уже длинные части РНК, что уже реализуемо.
    \end{itemize}

    \cimg{4.jpg}{0.6}
\end{frame}

\begin{frame}[t]{Фильтрация}
    \begin{itemize}
    \item
        Опять же можно убирать слишком короткие части из графа.
    \item
        Фильтрация по длине выравнивания.
    \item
        Фильтрация по качеству выравнивания.
    \end{itemize}
\end{frame}

\section{Итоги}
\begin{frame}[t]{Что получилось}
    Парные риды ДНК:
    \cimg{5.jpg}{1}
    РНК:
    \cimg{6.png}{0.5}
\end{frame}

\begin{frame}[t]{Чему мы научились}
    \begin{enumerate}
    \item
        Мы изучили основы биоинформатики.
    \item
        Научились применять различные программы, используемые в этой науке 
        (bowtie2, STAR, nucmer, SPAdes), на специально выделенном сервере.
    \item
        Разобрались с sam-форматом для хранения результатов выравнивания и 
        dot-форматом для визуализации графов.
    \item
        Получили опыт работы в команде.
    \end{enumerate}
\end{frame}

\begin{frame}[t]{Возможные пути развития}
    \begin{itemize}
    \item
        Дальнейшее исследование построения графа по ридам РНК.
    \item
        Объединение результатов обоих методов построения в один граф.
    \item
        Автоматизация подборки отсечений.
    \item
        Изучить другие идеи решения поставленной задачи, например, 
        \texttt{l\_rna\_scaffolder}.
    \end{itemize}
\end{frame}

\section{Спасибо за внимание}
\begin{frame}{Спасибо за внимание}
    \begin{center}
        Репозиторий: \\ \url{https://github.com/eshcherbin/spbau-bioinf-2016}
    \end{center}
\end{frame}
\end{document}
