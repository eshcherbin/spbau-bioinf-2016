\documentclass{beamer}

\usepackage{polyglossia}
\usepackage{fontspec}
\usepackage{nameref}
\usepackage{ifthen}

\usefonttheme{professionalfonts}
\usetheme{Antibes}
\useoutertheme{infolines_foot}
\setbeamercovered{transparent=20}

\usepackage[math-style=ISO,vargreek-shape=unicode]{unicode-math}

\setdefaultlanguage[spelling=modern,babelshorthands=true]{russian}
\setotherlanguage{english}

\defaultfontfeatures{Ligatures={TeX}}
\setmainfont{CMU Serif}
\setsansfont{CMU Sans Serif}
\setmonofont{CMU Typewriter Text}
\setmathfont{Latin Modern Math}
\AtBeginDocument{\renewcommand{\setminus}{\mathbin{\backslash}}}

\makeatletter
\newcommand*{\currentname}{\@currentlabelname}
\makeatother
\def\t{\texttt}

\newcommand{\cimg}[2]{%
	\begin{center}%
		\ifthenelse{\equal{#2}{}}{%
			\includegraphics[width=0.75\linewidth]{#1}
		}{%
			\includegraphics[width=#2\linewidth]{#1}
		}%
	\end{center}%
}

\title[Граф по ридам ДНК/РНК]{Построение графа связей геномных последовательностей по ридам ДНК/РНК}
\author{Черникова Ольга, Щербин Егор}
\institute{СПб АУ РАН}
\date{Весна 2016}

\begin{document}

\begin{frame}
	\titlepage
\end{frame}

\section{Постановка задачи}

\begin{frame}[t]{Проблема сборки генома}
    \begin{itemize}
    \item
        Одной из задач биоинформатики является задача сборки генома.
    \item
        Специальная программа (ассемблер) собирает геном организма по коротким
        его участкам (ридам).
    \item
        К сожалению, собрать геном полностью не получается. Вместо этого
        ассемблер выдаёт несколько достаточно больших его частей.
    \end{itemize}
\end{frame}

\begin{frame}[t]{Цель проекта}
    \begin{itemize}
    \item
        Наша задача "--- установить относительный порядок между этими частями.
    \item
        А именно, построить ориентированный граф, такой что если данные 
        указывают, что одна из них идёт перед второй, то проведём между ними 
        ребро.
    \end{itemize}
    %TODO: вставить картинку с большим графом
\end{frame}

\begin{frame}[t]{Подзадачи}
	\begin{enumerate}
    \item
        
	\end{enumerate}
\end{frame}

\section{Реализация}
\subsection{ДНК}

\begin{frame}[t]{Парные риды}
\end{frame}

\begin{frame}[t]{Отсечка}
\end{frame}

\subsection{РНК}

\begin{frame}[t]{Интроны и экзоны}
\end{frame}

\begin{frame}[t]{Проблема с частичным выравниванием}
\end{frame}

\begin{frame}[t]{Фильтрация}
\end{frame}

\section{Итоги}
\begin{frame}[t]{Что получилось}
\end{frame}

\begin{frame}[t]{Чему мы научились}
\end{frame}

\begin{frame}[t]{Что не успели}
\end{frame}

\section{Спасибо за внимание}
\begin{frame}{Спасибо за внимание}
    \begin{center}
        Репозиторий: \\ \url{https://github.com/eshcherbin/spbau-bioinf-2016}
    \end{center}
\end{frame}
\end{document}
